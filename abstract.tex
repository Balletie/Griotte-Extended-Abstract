%\newcommand{\CLASSINPUTbottomtextmargin}{20mm}
\documentclass[conference,a4paper]{IEEEtran}

%\usepackage{showframe}
\usepackage{graphicx}
\usepackage{cleveref}
\usepackage{footnote}
\usepackage{url}
\usepackage[inline]{enumitem}

\newcommand{\citeneeded}{\textbf{{[}citation needed{]}}}
\newcommand{\TC}{tangled commit}
\newcommand{\GCR}{Gerrit Code Review}
\newcommand{\Ep}{Epicea}
\newcommand{\Gr}{Griotte}
\newcommand{\code}[1]{\texttt{#1}}

\title{\Gr{}: Improving Code Review\\with Fine-Grained IDE Events}

\author{\IEEEauthorblockN{Skip~Lentz}\IEEEauthorblockA{EEMCS\\Delft
    University of Technology} \and
  \IEEEauthorblockN{Mart\'{i}n~Dias}\IEEEauthorblockA{RMoD\\INRIA
    Lille-Nord Europe}}

\begin{document}
\maketitle{}
\begin{abstract}
  Code review is a difficult process for several reasons, for example:
  \begin{enumerate*}[label=(\arabic*)]
  \item developers often create \textit{tangled commits},
  \item commits may touch many different parts of a project,
  \item many changes are shadowed,
  \item commit messages can be inaccurate or wrong and
  \item requesting a review for a change requires too much effort from
    the perspective of the developer.
  \end{enumerate*}

  This work aims to propose a solution to these problems by exploiting
  the information provided by fine-grained IDE events.

  To put this solution in practice, we will develop a code review tool
  named \textit{\Gr} in the \textit{Pharo} environment.
\end{abstract}
\begin{IEEEkeywords}
  code review, fine-grained IDE events, \Ep, \Gr, Pharo
\end{IEEEkeywords}

\section{Problem Description}
\label{sec:problem-description}
Modern Code Review (MCR) is an important mechanism for quality
assurance in software evolution: it provides feedback and avoids
introducing bugs. However, it is a difficult task for a reviewer to
perform. This is due to multiple reasons, such as:

\paragraph{Tangled commit}

A \TC{} is a commit which is the result of multiple, mutually
unrelated changes. A commit could consist of a refactoring
\cite{Fowl99a}, some formatting changes, and a change to the body of a
method (the latter being the change for which the commit is actually
meant).

The process of understanding a change is difficult on its own. In the
case of a tangled commit, the reviewer in addition has to understand
several unrelated changes at the same time.

Developers often create tangled commits \citeneeded{}.

\paragraph{Line-based changes}

Commits may have changes scattered across many different parts of a
project. A refactoring of a method name is a good example of this,
since all places where the method is referenced are also
changed. However, even though they belong together, they are not shown
together within the changes browser of the code review tool. For this
reason, we say that these review tools are \textit{line-based}.

Thus, the reviewer has to analyze the changes line-by-line to conclude
that it was a refactoring. They are not grouped together.

\paragraph{Shadowed changes}

Negara \textit{et al.} found that 37\% of changes are shadowed and
thus do not reach the Version Control System
(VCS)\cite{Nega12a}\textbf{{[}TODO: correct way to cite?{]}}. This can
be important information for the reviewer to find out how the
developer arrived at their solution.

\paragraph{Wrong or lack of commit descriptions}

Commit descriptions can help the reviewer in understanding the reason
of a change. Without the description, the reason for a change is
lost. Furthermore, a wrong description has the potential to mislead
the reviewer.

Commit messages are often inaccurate or wrong\citeneeded{}.

\paragraph{Effort in sharing a change}

\textbf{{[}TODO: requires expanding{]}}requesting a review for a change requires too much effort from
  the perspective of the developer.

\section{Proposed Solution}
\label{sec:proposed-solution}
We propose a solution to this problem by introducing the concept of
groups. A group consists of one or more code changes which are
mutually related, accompanied with a descriptive label. An example
would be the refactoring of a method's name from \code{foo} to
\code{bar}. By grouping changes, we change the perspective from a
\textit{line}-based review tool, to a \textit{change}-based review
tool.

In practice, these groups can be created automatically (as is the case
for refactoring changes), or manually using specially designed tools.

To be able to group code changes, we need fine-grained
information. This information is recorded by --- and collected from
--- \textit{\Ep}, which will be discussed in the next section.

\section{First-Class Code Changes from \Ep}
\label{sec:first-class-ide}
\Ep\ is a Pharo project providing a model for first-class code changes
and related IDE tools. \Ep\ records code changes during development,
allowing us to analyze a more fine-grained history of the
codebase. This is in contrast with the code change information
available from a VCS repository, which is much more
coarse-grained. The model of \Ep\ can be divided into two parts:
\textit{high-level} and \textit{low-level} code changes.

Low-level code changes are additions, deletions or modifications of
program components such as classes or methods. Examples of these code
changes are a method modification, a class addition, etc.

Secondly, \Ep\ records IDE events such as a test run (and its
outcome), refactorings, and loading a version from the VCS. These are
known as high-level code changes.

\Ep\ records this data by means of the \textit{\Ep\ Monitor}, which
listens to events in Pharo as they happen, and converts them to the
first-class code change model objects described above. These objects
are then stored and persisted in a log.

\section{Forming Groups using First-Class Code Changes}
\label{sec:code-review-using}
There are different strategies we can employ to form groups of code
changes. In the simplest sense, they can be formed automatically or
manually. In this section we discuss which strategy --- automatic or
manual --- is the right one for various kinds of groups.

\subsection{Automatically forming groups}
\label{sec:autom-form-groups}
The automatic approach is likely the most suitable for code changes
done by a refactoring. Since refactorings are provided with rich
information in the Epicea model, forming this code change as a group
and creating a descriptive label should be straightforward.

Another approach is comparing the \textit{abstract syntax tree}\ (AST)
to detect whether or not the code change entails a change of behavior
in the program. While the detection of this is straightforward,
creating a descriptive label is less so. For example, both the
addition of a comment and a formatting in the body of a method cause
no change in the AST. Thus, in this case, the labels of the groups
become less descriptive. However, even a non-descriptive label such as
``No change in behavior'' assists the reviewer in understanding the
code more than there being no label at all.

We foresee a challenge in this approach; namely that many changes are
shadowed or undone and thus do not reach the
VCS\cite{Nega12a}. Detecting this is not trivial, but necessary to
reduce the unneeded groups in the output.

\subsection{Manually forming groups}
\label{sec:manu-form-groups}
When groups are formed manually, we have the benefit of being able to
create a very precise label for the group. An obvious downside is that
this requires us to count on the assistance of the developer. A
separate tool for this could therefore prove useful.

An example of such a tool is the \textit{\Ep\ Task
  Clusterer}\cite{Dias15a}, which provides the developer with an
intuitive user interface to group ``tasks''. In this context, a task
refers to the resulting code of resolving an issue or bug-report.

\section{Implementation}
\label{sec:implementation}
We aim to implement a code review tool named \textit{\Gr} for Pharo
which employs the concept of groups. This section aims to present the
implementation of the tool. An important part of the tool is that it
leverages existing services providing code review features such as
\textit{GitHub} or \textit{Gerrit}. See \cref{fig:diagram} for a
simplified diagram of the architecture.
\begin{figure}[t]
  \centering{\resizebox{\linewidth}{!}{\includegraphics{idea.png}}}
  \caption{Simplified architecture diagram}
  \label{fig:diagram}
\end{figure}

\subsection{Leveraging existing services}
\label{sec:lever-exist-serv}
A key part of our implementation will be the use of existing
services. We discuss the advantages and disadvantages, and how this
approach is to be implemented.

An advantage is that this allows us to alleviate ourselves from part
of the maintenance of the code review tool. This includes both server
maintenance and server-side code maintenance. Furthermore, we get
security and additional functionality for free.

A disadvantage is that we need to sacrifice some of our own
flexibility, and rely on the flexibility of the API's of external
services like GitHub. In a research project this might seem less
ideal, as one would like to be free to explore different
options. However, we feel that we have found the middle way between a
pragmatic solution and a more research oriented solution.

The idea is that the special features which use the concept of groups
are to be accessible only client-side, i.e. from within Pharo. The
information necessary for the groups can be stored as metadata on the
external service. For example, in the case of git-based services such
as GitHub or Gerrit, we can use the
\code{git-notes}\footnote{Documentation:
  \url{http://git-scm.com/docs/git-notes}} model to store metadata on
commits.

\section{Conclusion}
\label{sec:conclusion}
To conclude and summarize, we presented the problems with Modern Code
Review, namely that it does not handle tangled commits, and uses a
line-based approach for review.

We proposed an approach aiming to solve these problems, namely a
change-based form of code review using the concept of groups. The
groups are formed using fine-grained code changes provided by \Ep. We
divided the forming of groups into two categories, automatic and manual.

Furthermore, we presented our proposal for an implementation of a code
review tool using the concept of groups, named \Gr. A key idea of our
implementation is the use of existing external services such as GitHub
and Gerrit. We discussed the benefits and limitations of this
approach.

\bibliographystyle{IEEEtran}
\bibliography{IEEEabrv,rmod,others}
\end{document}

%%% Local Variables:
%%% mode: latex
%%% TeX-master: t
%%% End:
